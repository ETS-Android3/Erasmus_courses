% \section{Introduction}
% \section{Software and Hardware Specifications}
% \subsection{Software Selection}
% \subsection{Hardware requirements}
% \subsubsection{General Specifications}
% \subsubsection{Detailed Technical Specifications}

% \section{System Overview}
% \subsection{System Information} 


%%%%%%%%%%%%%%%%%%%%%%%%%%%%%%%%%%%%%%%%%%


% \section{Database Design and Functionalities}
% \subsection{Design & Functional Support}
% \subsection{User Management}

% Aplikacja webowa będzie składać się z TRZECH podstawowych typów użytkowników (uczeń, korepetytor, ADMINISTRATOR) 
% Funkcje produktu 

% System umożliwia przeglądanie profili korepetytorów kontakt z nimi, pobieranie/udzielanie korepetycji online. Aplikacja powinna umożliwiać korepetytorowi edycję danych o oferowanych przez niego usługach edukacyjnych, zamieszczenie danych kontaktowych i regionu, w którym gotowy jest udzielać usług fizycznie. 
% Profil ucznia powinien umożliwiać przeglądanie profili korepetytorów oraz filtrowanie bazy danych w celu znalezienia właściwego usługodawcy. Nawiązanie kontaktu z korepetytorem                                      i wystawienie opinii, w formie pisemnego komentarza oraz ratingu (gwiazdek).  Dodatkowo aplikacja będzie udostępniać uczniom informacje o dostępności korepetytorów na stronie (dostępny, niedostępny). 
% Korepetytor będzie dysponował pakietem rozwiązań i narzędzie do przeprowadzenia zdalnych zajęć, które będzie udostępniał zainteresowanemu uczniowi, na czas zajęć. 
% Zarówno korepetytor jak i uczeń będzie mógł zablokować możliwość wzajemnego kontaktu. Strona będzie posiadała również system raportowania o złamaniu podstawowych norm zachowania, które będą rozpatrywane administracyjnie i odpowiednio sankcjonowane.
% Możliwość wystawiania ogłoszeń będą miały również firmy edukacyjne (np. szkoły językowe). Profile firm edukacyjnych będą dysponowały takim samym pakietem rozwiązań jak profile korepetytorów, jednak na liście ogłoszeń wyróżniać się one będą specjalnym graficznym oznaczeniem. 

% Charakterystyka użytkowników 

%          W tym podpunkcie omówione zostały funkcje profili dostępnych na aplikacji. Wyróżnia się dwa podstawowe typy profili: uczeń, korepetytor. 
%          Uczeń będzie użytkownikiem posiadającym dostęp do bazy korepetytorów, których będzie mógł filtrować pod względem nauczanych przez nich przedmiotów, poziomu nauczania, miejscowości                  w której prowadzą swoją działalność, stawki godzinowej którą biorą, opinii innych użytkowników (gwiazdek) i możliwości prowadzenia nauczania zdalnego. Ponadto uczeń będzie widział którzy korepetytorzy aktualnie są dostępni na portalu i będzie mógł inicjować z nimi rozmowę na czasie. 
%          Osoba zakładająca na portalu profil ucznia będzie musiała obowiązkowo podać swoje imię            i wiek, dodatkowo będzie mogła wstawić zdjęcie profilowe i podać poziom edukacji na którym się znajduje (szkoła podstawowa, szkoła średnia, szkoła wyższa).  Informacje te będą widoczne dla korepetytorów. 
%            Korepetytor nie będzie miał dostępu do bazy uczniów i tym samym nie będzie mógł inicjować rozmowy na czacie z uczniem. Dopiero po napisaniu pierwszej wiadomości przez ucznia stanie się dla niego widoczny i będzie możliwa interakcja na czasie. 
%          Osoba zakładająca na portalu profil korepetytora będzie musiała podać swoje imię, wiek, przedmioty które chce nauczać, określić poziom którego nauczanie jest zainteresowany, stawkę godzinową którą chce pobierać za lekcje, oraz określić czy jest zainteresowany prowadzeniem korepetycji zdalnych. Dodatkowo możliwe do podania będzie numer telefonu, mail, zdjęcie profilowe    i krótki opis. Wszystkie te informacje będą widoczne dla ucznia po wejściu na profil korepetytora. 
%                 Uczeń w bazie danych korepetytorów będzie widział zdjęcie profilowe korepetytora, jego imię, ilość gwiazdek które uzyskał od swoich uczniów i miniaturki przedmiotów które naucza.                     Po wejściu w profil użytkownika uczeń zobaczy bardziej szczegółowe informacje i opinie innych użytkowników na temat korepetytora.

