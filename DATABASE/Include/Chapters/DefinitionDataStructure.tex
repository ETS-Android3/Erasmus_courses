\section{Defining the data structure}
Definition of data structure and Rest points was done in 
OpenAPI Specification - Version 3.0.0. Full file is in the \hyperref[sec:appendix_a]{Appendix A}.

Basic data model of \textit{UserBasicData} can be seen at the listing below.
\lstinputlisting[firstnumber=565, firstline=565, lastline=589, breaklines=true, numbers=left, stepnumber=1]{Include/SourceCode/openapi.yaml}


Basic data model are often part of the more complex structure like definition of \textit{User} structure.
\lstinputlisting[firstnumber=559, firstline=559, lastline=564, breaklines=true, numbers=left, stepnumber=1]{Include/SourceCode/openapi.yaml}

Some of structures can be used to modify data in the database or get data from the database. Manipulating database was done by RestAPI. Example of usage data structures is described in GET, PUT and DELETE requests. GET is used to get data from the database, PUT is used to update data existing in database and DELETE is used to remove data object from database.
\lstinputlisting[firstline=76, lastline=118, firstnumber=76, breaklines=true, numbers=left, stepnumber=1]{Include/SourceCode/openapi.yaml}

POST method is used to creating not existing data object in database. Rest point \textit{/api/v1/signup} allows new user to create new account. 
\lstinputlisting[firstnumber=16, firstline=16, lastline=34, breaklines=true, numbers=left, stepnumber=1]{Include/SourceCode/openapi.yaml}

Security and authentication of the rest points was done with usage of the bearerAuth. The bearer token is a cryptic string, generated by the server in response to a login request. The client must send this token in the Authorization header when making requests to protected resources. Cryptic string contains user data. They are used to setup frontend according to user constraints and preferences. If string containing user data would be changed then bearer token would change and there would be not possible to authenticate these token by server.
\lstinputlisting[firstnumber=35, firstline=35, lastline=75, breaklines=true, numbers=left, stepnumber=1]{Include/SourceCode/openapi.yaml}





% \begin{lstlisting}
% \end{lstlisting}
