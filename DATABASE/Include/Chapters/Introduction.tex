\section{Introduction}

% Easy Learn Community jest aplikacją webową która łączy korepetytorów i uczniów oraz umożliwia udzielanie i pobieranie korepetycji bez konieczności wychodzenia z domu. 

Easy Learn Community is a web application that connects tutors and students and allows the user to give and receive tutoring while staying at home.


% Aplikacja będzie oferować bazę danych zarejestrowanych korepetytorów wraz z podstawowymi informacjami na ich temat. Z poziomu przeglądarki internetowej osoby zainteresowane pobieraniem korepetycji będą miały możliwość przeglądania ofert bez wcześniejszego zalogowania, jak również kontaktu z potencjalnym korepetytorem. Historia rozmowy zapisywana będzie w aplikacji po wcześniejszym zarejestrowaniu się przez zainteresowanego (profil uczeń). Aplikacja w swoim zakresie będzie obejmowała niezbędne narzędzia do udzielania korepetycji online. 

The application will offer a database of registered tutors together with basic information about them. From the level of a web browser, people interested in receiving tutoring will be able to browse the offers without prior logging in, as well as contact a potential tutor. The conversation history will be recorded in the application after the interested party has registered (student profile). The application will include the necessary tools for online tutoring. 


% \subsection{Purpose}
\subsection{Scope, Approach and Methods}
Scope of the project includes:
\begin{itemize}
    \item Create definition of the database containing all fields needed to run the application.
    \item Create basement of the application working without implemented video and chat communication service.
\end{itemize}

Application will be based on the Ryan's Chenkie react app named \href{https://github.com/chenkie/orbit}{"ReactSecurity - Orbit"}(MIT licence).

The database will be developed in MongoDB technology that is NoSQL database technology often used in webapps.


\subsection{System Overview}

% Ta sekcja zawiera przegląd całej aplikacji webowej. Aplikacja zostanie wyjaśniona i omówiona w kontekście jej zastosowania.

This section provides an overview of the entire web application. The application will be explained and discussed in the context of its application.

% Aplikacja zakłada główne funkcjonalności systemu:
% * Ogłaszanie się korepetytorów poprzez podanie swoich danych kontaktowych widocznych dla wszystkich użytkowników systemu
% * Prowadzenie korepetycji online przez dedykowaną aplikację do prowadzenia rozmów online
% * Komunikacja się z dowolnym z korepetytorów poprzez czat
% * Widoczna dostępność online korepetytorów  oraz możliwe natychmiastowe udzielanie korepetycji przez dostępnych korepetytorów
% * Sortowanie korepetytoró na listach jest ściśle zależne od rankingu korepetytorów, który jest ustalany na podstawie opinii


The application assumes the main system functionalities:
\begin{itemize}
    \item Announcement of tutors by providing their contact details visible to all system users
    \item Tutoring online through a dedicated online chat application
    \item Communication with any of the tutors via chat
    \item Visible online availability of tutors and possible immediate tutoring by available tutors
    \item The sorting of the tutors on the lists is closely linked to the ranking of tutors, which is determined by the opinion
\end{itemize}


%%%%%%%%%%%%%%%%%%%%%%%%%%%%%%%%%%%%%%%%%%


% \subsection{Acronyms and Abbreviations}
% Użytkownik – osoba korzystająca z aplikacji webowej
% Korepetytor – osoba zarejestrowana na aplikacji oferująca udzielanie korepetycji
% Uczeń – osoba zarejestrowana na aplikacji w celu pobierania korepetycji
% Opiekun – osoba na którą zarejestrowane jest konto w przypadku ucznia poniżej 16 roku życia
% Firma edukacyjna
% Profil korepetytor – profil założony przez korepetytora oferujący narzędzia do udzielania  korepetycji
% Profil uczeń – Profil umożliwiający kontakt z korepetytorem 
% Profil firma
% Ogólny opis 



%%%%%%%%%%%%%%%%%%%%%%%%%%%%%%%%%%%%%%%%%%

